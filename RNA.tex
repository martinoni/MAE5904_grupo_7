\documentclass{article}
\usepackage[utf8]{inputenc}

\title{RNA}
\author{lucas.bortolucci }
\date{December 2020}

\begin{document}

\maketitle

\section{Redes neurais artificiais}

O modelo de redes neurais artificiais foi implementado em python com a biblioteca Keras (Tensorflow), com o acompanhamento do treino feito pelo Comet.ml. Em sua arquitetura, tratando-se de uma Rede Neural totalmente conectada, possui uma hidden layer (com 8 unidades), 137 parâmetros e função de ativação sigmoid:

\begin{equation}
\sigma(x) = \frac{1}{1 + e^x}.
\end{equation}

Outras caracteristicas da rede e de configuração de otimização são:

\begin{itemize}
	\item  Função de custo
       	\begin{itemize}
     		\item  Binnary Cross-Entropy
     		\item $H_P(q) = - \frac{1}{N} \displaystyle\sum_{i=1}^{N} y_i \cdot log(p(y_i)) + (1 - y_i) \cdot log(1 - p(y_i))$
       		\end{itemize}
	\item  Otimizador
       	\begin{itemize}
     		\item Adam (Padrão)
       		\end{itemize}
	\item  Batch size
       	\begin{itemize}
     		\item  1024 indivíduos
       		\end{itemize}
	\item  Epochs
       	\begin{itemize}
     		\item  100
       		\end{itemize}
    \item  Método de escolha do ponto de corte
       	\begin{itemize}
     		\item  Youden's J statistic
       		\end{itemize}
\end{itemize}

Com essas características os resultados obtidos foram:

\begin{itemize}
	\item  Treino
       	\begin{itemize}
       	    \item AUC: 0.67
     		\item Acurácia: 0.52
     		\item Sensibilidade: 0.81
     		\item Especificidade: 0.40
     	\end{itemize}
     \item  Treino
       	\begin{itemize}
       	    \item Acurácia: 0.55
     		\item Sensibilidade: 0.83
     		\item Especificidade: 0.36
       	\end{itemize}
\end{itemize}

Como esses resultados foram semelhantes aos do modelo de reressão logística, que possui uma melhor interpretabilidade e menor complexidade de implementação (e tempo de ajuste), acabamos por não escolher as redes neurais artificiais como modelo final do projeto.

\end{document}
